%%
%% This is file `sample-acmsmall-submission.tex',
%% generated with the docstrip utility.
%%
%% The original source files were:
%%
%% samples.dtx  (with options: `acmsmall-submission')
%% 
%% IMPORTANT NOTICE:
%% 
%% For the copyright see the source file.
%% 
%% Any modified versions of this file must be renamed
%% with new filenames distinct from sample-acmsmall-submission.tex.
%% 
%% For distribution of the original source see the terms
%% for copying and modification in the file samples.dtx.
%% 
%% This generated file may be distributed as long as the
%% original source files, as listed above, are part of the
%% same distribution. (The sources need not necessarily be
%% in the same archive or directory.)
%%
%%
%% Commands for TeXCount
%TC:macro \cite [option:text,text]
%TC:macro \citep [option:text,text]
%TC:macro \citet [option:text,text]
%TC:envir table 0 1
%TC:envir table* 0 1
%TC:envir tabular [ignore] word
%TC:envir displaymath 0 word
%TC:envir math 0 word
%TC:envir comment 0 0
%%
%%
%% The first command in your LaTeX source must be the \documentclass
%% command.
%%
%% For submission and review of your manuscript please change the
%% command to \documentclass[manuscript, screen, review]{acmart}.
%%
%% When submitting camera ready or to TAPS, please change the command
%% to \documentclass[sigconf]{acmart} or whichever template is required
%% for your publication.
%%
%%
\documentclass[acmsmall,screen,review]{acmart}

% \usepackage[style=authoryear, useprefix = false]{biblatex}

%%
%% \BibTeX command to typeset BibTeX logo in the docs
\AtBeginDocument{%
  \providecommand\BibTeX{{%
    Bib\TeX}}}

%% Rights management information.  This information is sent to you
%% when you complete the rights form.  These commands have SAMPLE
%% values in them; it is your responsibility as an author to replace
%% the commands and values with those provided to you when you
%% complete the rights form.
% \setcopyright{acmlicensed}
% \copyrightyear{2018}
% \acmYear{2018}
% \acmDOI{XXXXXXX.XXXXXXX}


%%
%% These commands are for a JOURNAL article.
% \acmJournal{JACM}
% \acmVolume{37}
% \acmNumber{4}
% \acmArticle{111}
% \acmMonth{8}

%%
%% Submission ID.
%% Use this when submitting an article to a sponsored event. You'll
%% receive a unique submission ID from the organizers
%% of the event, and this ID should be used as the parameter to this command.
%%\acmSubmissionID{123-A56-BU3}

%%
%% For managing citations, it is recommended to use bibliography
%% files in BibTeX format.
%%
%% You can then either use BibTeX with the ACM-Reference-Format style,
%% or BibLaTeX with the acmnumeric or acmauthoryear sytles, that include
%% support for advanced citation of software artefact from the
%% biblatex-software package, also separately available on CTAN.
%%
%% Look at the sample-*-biblatex.tex files for templates showcasing
%% the biblatex styles.
%%

%%
%% The majority of ACM publications use numbered citations and
%% references.  The command \citestyle{authoryear} switches to the
%% "author year" style.
%%
%% If you are preparing content for an event
%% sponsored by ACM SIGGRAPH, you must use the "author year" style of
%% citations and references.
%% Uncommenting
%% the next command will enable that style.
%%\citestyle{acmauthoryear}


%%
%% end of the preamble, start of the body of the document source.
\begin{document}

%%
%% The "title" command has an optional parameter,
%% allowing the author to define a "short title" to be used in page headers.
\title{FAT-Pointer based range addresses}

%%
%% The "author" command and its associated commands are used to define
%% the authors and their affiliations.
%% Of note is the shared affiliation of the first two authors, and the
%% "authornote" and "authornotemark" commands
%% used to denote shared contribution to the research.
\author{Akilan Selvacoumar}
% \authornote{Both authors contributed equally to this research.}
\email{as251@hw.ac.uk}
\orcid{1234-5678-9012}
% \author{G.K.M. Tobin}
% \authornotemark[1]
% \email{webmaster@marysville-ohio.com}
\affiliation{%
  \institution{Heriot Watt University}
  % \streetaddress{P.O. Box 1212}
  % \city{Dublin}
  % \state{Ohio}
  \country{UK}
  % \postcode{43017-6221}
}

% \author{Lars Th{\o}rv{\"a}ld}
% \affiliation{%
%   \institution{The Th{\o}rv{\"a}ld Group}
%   \streetaddress{1 Th{\o}rv{\"a}ld Circle}
%   \city{Hekla}
%   \country{Iceland}}
% \email{larst@affiliation.org}

% \author{Valerie B\'eranger}
% \affiliation{%
%   \institution{Inria Paris-Rocquencourt}
%   \city{Rocquencourt}
%   \country{France}
% }

% \author{Aparna Patel}
% \affiliation{%
%  \institution{Rajiv Gandhi University}
%  \streetaddress{Rono-Hills}
%  \city{Doimukh}
%  \state{Arunachal Pradesh}
%  \country{India}}

% \author{Huifen Chan}
% \affiliation{%
%   \institution{Tsinghua University}
%   \streetaddress{30 Shuangqing Rd}
%   \city{Haidian Qu}
%   \state{Beijing Shi}
%   \country{China}}

% \author{Charles Palmer}
% \affiliation{%
%   \institution{Palmer Research Laboratories}
%   \streetaddress{8600 Datapoint Drive}
%   \city{San Antonio}
%   \state{Texas}
%   \country{USA}
%   \postcode{78229}}
% \email{cpalmer@prl.com}

% \author{John Smith}
% \affiliation{%
%   \institution{The Th{\o}rv{\"a}ld Group}
%   \streetaddress{1 Th{\o}rv{\"a}ld Circle}
%   \city{Hekla}
%   \country{Iceland}}
% \email{jsmith@affiliation.org}

% \author{Julius P. Kumquat}
% \affiliation{%
%   \institution{The Kumquat Consortium}
%   \city{New York}
%   \country{USA}}
% \email{jpkumquat@consortium.net}

%%
%% By default, the full list of authors will be used in the page
%% headers. Often, this list is too long, and will overlap
%% other information printed in the page headers. This command allows
%% the author to define a more concise list
%% of authors' names for this purpose.
% \renewcommand{\shortauthors}{Trovato et al.}

%%
%% The abstract is a short summary of the work to be presented in the
%% article.
\begin{abstract}
  % Draft 1
  % The gap between application workloads and capacity of TLB is on the 
  % rise. Previous work exploits this by physically contiguous 
  % memory. Capability-based addressing in the CHERI architecture is 
  % designed to improve hardware-level system security.
  % These security mechanisms designed in place can also 
  % behave as accelerators to standard user-space memory allocators. 
  % The property of CHERI which is to use FAT-Pointers to store
  % the upper and lower bounds can also be represented as ranges of 
  % memory addresses which are physically contiguous. 
  % <Talking about the results>.
  % Draft 2
  The increasing disparity between application workloads and the capacity of Translation Lookaside Buffers (TLB) 
  has prompted researchers to explore innovative solutions to mitigate this gap. One such approach involves 
  leveraging physically contiguous memory to optimize TLB utilization. Concurrently, advancements in hardware-level 
  system security, exemplified by the Capability Hardware Enhanced RISC Instructions (CHERI) architecture, offer 
  additional opportunities for improving memory management and security.

 CHERI introduces capability-based addressing, a novel approach that enhances system security by associating capabilities 
 with memory pointers. These capabilities restrict access to memory regions, thereby fortifying the system against various 
 security threats. Importantly, the mechanisms implemented in CHERI for enforcing memory protection can also serve as 
 accelerators for standard user-space memory allocators. By leveraging capability-based addressing, memory 
 allocators can efficiently manage memory resources while ensuring robust security measures are in place.
\end{abstract}

%%
%% The code below is generated by the tool at http://dl.acm.org/ccs.cfm.
%% Please copy and paste the code instead of the example below.
%%
% \begin{CCSXML}
% <ccs2012>
%  <concept>
%   <concept_id>00000000.0000000.0000000</concept_id>
%   <concept_desc>Do Not Use This Code, Generate the Correct Terms for Your Paper</concept_desc>
%   <concept_significance>500</concept_significance>
%  </concept>
%  <concept>
%   <concept_id>00000000.00000000.00000000</concept_id>
%   <concept_desc>Do Not Use This Code, Generate the Correct Terms for Your Paper</concept_desc>
%   <concept_significance>300</concept_significance>
%  </concept>
%  <concept>
%   <concept_id>00000000.00000000.00000000</concept_id>
%   <concept_desc>Do Not Use This Code, Generate the Correct Terms for Your Paper</concept_desc>
%   <concept_significance>100</concept_significance>
%  </concept>
%  <concept>
%   <concept_id>00000000.00000000.00000000</concept_id>
%   <concept_desc>Do Not Use This Code, Generate the Correct Terms for Your Paper</concept_desc>
%   <concept_significance>100</concept_significance>
%  </concept>
% </ccs2012>
% \end{CCSXML}

% \ccsdesc[500]{Do Not Use This Code~Generate the Correct Terms for Your Paper}
% \ccsdesc[300]{Do Not Use This Code~Generate the Correct Terms for Your Paper}
% \ccsdesc{Do Not Use This Code~Generate the Correct Terms for Your Paper}
% \ccsdesc[100]{Do Not Use This Code~Generate the Correct Terms for Your Paper}

%%
%% Keywords. The author(s) should pick words that accurately describe
%% the work being presented. Separate the keywords with commas.
% \keywords{Do, Not, Us, This, Code, Put, the, Correct, Terms, for,
%   Your, Paper}

% \received{20 February 2007}
% \received[revised]{12 March 2009}
% \received[accepted]{5 June 2009}

% Sample citation
% \cite{ad-wood-2003}

%%
%% This command processes the author and affiliation and title
%% information and builds the first part of the formatted document.
\maketitle

\section{Introduction}
In the dynamic landscape of computing, the pursuit of optimal performance is a constant endeavor, 
especially as applications evolve to handle increasingly complex workloads. 
One critical aspect influencing performance is memory management, where efficient 
utilization of resources is paramount. Translation Lookaside Buffers (TLBs) play a 
pivotal role in this regard, expediting memory access by storing recently accessed memory translations.
However, as applications grow in size and complexity, the capacity of TLBs often struggles to 
keep pace, leading to performance bottlenecks. To address this challenge, researchers have 
turned to innovative solutions, one of which involves harnessing the benefits of huge pages.
Huge pages, also known as large pages, allow for the allocation of memory in significantly 
larger chunks compared to traditional small pages. By reducing the number of TLB entries 
needed to access a given amount of memory, huge pages offer a potential avenue for optimizing 
TLB utilization and thereby enhancing overall system performance.

Simultaneously, advancements in hardware-level security, such as the Capability Hardware 
Enhanced RISC Instructions (CHERI) architecture, present additional opportunities for 
performance enhancement. CHERI's capability-based addressing approach not only strengthens 
system security by tightly controlling memory access but also provides avenues for 
accelerating memory management operations.

In this context, the integration of huge pages into memory management 
strategies alongside capability-based addressing in architectures like 
CHERI offers a compelling synergy. By optimizing TLB utilization through the 
utilization of huge pages and leveraging the security features of capability-based addressing, 
significant performance improvements can be realized. This approach not only enhances 
system security but also accelerates memory access.

% TODO: 
% - Add references for FlexPointer, Range Memory Mapping (RMM), Direct Segment and Huge Pages
\subsection{TLB based approaches}
Efficient memory management, particularly in the context of 
Translation Lookaside Buffer (TLB) optimization, has been a focal point of 
research and development within computer architecture. Various techniques have been 
proposed to mitigate TLB-related bottlenecks and improve overall system performance.

\subsubsection{Huge Pages:}
This is used to map a very large region of memory to a 
single entry. This small/large region of memory is physically
contiguous. Most implementations of huge pages \cite{panwar_hawkeye_2019} are size
aligned, For example for the x86 architecture the huge pages 
size are 4KB, 2MB and 1GB pages. 
% \subsection{Tailored page sizes}
% TODO later
% \subsection{TLB coalescing}
% This leverages the default OS allocator behavior to pack
% multiple PTEs into a single TLB entry.

\subsubsection{Segment:}
A segment can be viewed as mapping between contiguous virtual
memory and contiguous physical memory. The property of a 
segment allows it to be larger than a page. Direct Segment allows the user to set a single segment
for an application. Two registers are added to mark the start
and end of the segment. Any virtual address within this region
can be translated by adding the fixed offset between the virtual
and physical address.

\subsubsection{Range Memory Mapping (RMM):}
RMM\cite{karakostas_redundant_2015} introduces the concept of adding an additional range table.
For large allocations RMM eagerly allocates contiguous physical pages.
The following allocations creates large memory ranges that are
both virtually and physically contiguous. RMM builds on the concept
of Direct segment by adding offset to translate a virtual address 
to physical address. RMM compares address with range boundaries 
to decide which range it belongs to. RMM queries the range table 
ofter an L1 TLB miss.

\subsubsection{FlexPointer:}
FlexPointer\cite{chen_flexpointer_2023} is based on the RMM\cite{karakostas_redundant_2015} paper. FlexPointer
does eagerly allocate pages which are physically contiguous and stores the ID to translate a virtual address 
to physical address on the remaining unused bits on the 64 bit virtual address. 
The paper contribution mentions shifting the TLB lookup to an earlier stage to improve
latency of accessing the TLB entries. FlexPointer immediate queries the 
range TLB for translations rather than the RMM paper which waits for the L1 TLB miss.

\subsection{CHERI}
A capability functions as a token that provides the holder with the authority to 
execute specific actions. By carefully managing who possesses these capabilities, 
it is possible to enforce security measures, such as ensuring that a particular
section of software can only read from a designated range of memory addresses.

The concept of capabilities has a rich history in computer science, tracing 
back to early systems designed to enhance security and manage resources effectively.
For instance, foundational work discussed in sources like [7] offers a comprehensive 
narrative on the evolution of capability architectures. This historical perspective 
can be further enriched by incorporating insights from more recent studies, 
such as those found in [11].

CHERI (Capability Hardware Enhanced RISC Instructions) represents a modern embodiment of this 
long-standing idea. It introduces more granular control over permissions, allowing for finer 
distinctions in what actions can be performed and by whom. Moreover, CHERI is designed to integrate 
seamlessly with contemporary processor instruction sets, ensuring that these advanced security 
features can be implemented on modern hardware platforms. This adaptation not only revitalizes 
the capability concept but also significantly enhances its applicability and effectiveness 
in current computing environments.


\section{FAT-Pointer based range addresses}
% This experiment exploits the FAT pointer to enable faster

% TODO: 
% - Add references for FlexPointer, Range Memory Mapping (RMM) and Direct segment
% \subsection{Overview}

FAT-Pointers, combined with the capabilities of the CHERI (Capability Hardware Enhanced RISC Instructions) 
architecture, introduce robust memory safety and security features by incorporating additional metadata 
with memory pointers. This enhanced architecture utilizes concepts such as FlexPointer, 
Range Memory Mapping (RMM) to manage memory effectively.

Range addresses play a pivotal role within this framework, defining memory 
regions bounded by a starting address (Upper) and an ending address (Lower). 
These range addresses are encoded within FAT-pointers, allowing for precise 
control over memory regions.

The functionality of ranges encompasses several key aspects:
\begin{itemize}
\item \textbf{Creation of Physically Contiguous Memory Ranges}:
By defining memory regions that are physically contiguous, systems can 
achieve optimal memory access patterns, enhancing performance and efficiency.
\item \textbf{Encoding Ranges as Bounds to the Pointer}:
Integrating range bounds directly into FAT-pointers enables the architecture 
to enforce memory access restrictions at the pointer level thus allowing 
tracking of memory ranges on a pointer level.
\item \textbf{Instrumenting Block-Based Allocators with Physically Contiguous Memory}:
The integration of range-based memory concepts into memory allocation systems, such as block-based 
allocators, facilitates the efficient management and utilization of physically contiguous memory blocks, 
mitigating issues related to memory fragmentation.
\end{itemize}

% \begin{minipage}[t]{0.4\linewidth}
\begin{figure}[h]
  \includegraphics[width=0.8\textwidth]{diagrams/HighOverviewArchitecture24.png}
  \caption{High overview architecture}
  \label{fig:HighOverviewArchitecture}
% \end{minipage}
\end{figure}

% The figure above demonstrates the approach taken for using CHERI 
% 128bit FAT pointer scheme to allow a blocked based behavior on 
% on physically contiguous memory which is on the right against the 
% regular mmap approach which further elaborated in section (). The 
% green highlighted refers to the excess space available between 
% the 48th and 64th bit which can be used to store more meta-data which 
% is elaborated in the future work section().


Figure \ref{fig:HighOverviewArchitecture} illustrates the methodology employed to leverage the CHERI 
128-bit FAT-pointer scheme for facilitating block-based memory management
 on physically contiguous memory, which is depicted on the right side of the figure. 
 This technique contrasts with the conventional mmap approach, the details of which 
 are elaborated in section ("To be added later").

In figure \ref{fig:HighOverviewArchitecture}, the green-highlighted section marks the unused space between the 48th and 64th bits
within the FAT-pointer. This area of unused bits presents an opportunity to store additional metadata,
potentially enhancing the capabilities of the memory management system. The potential applications of 
this extra space are discussed in the future work section ("To be added later"), 
where we explore how this additional metadata storage could be used to further optimize memory allocation.

% By employing the CHERI 128-bit FAT-pointer scheme, the approach depicted aims to streamline the management of memory blocks, ensuring they are physically contiguous. This is in contrast to the traditional mmap approach, which typically does not guarantee physical contiguity of allocated memory regions.


\subsection{Range creation and huge pages}
% \begin{minipage}[t]{0.4\linewidth}
\begin{figure}[h]
  \includegraphics[width=0.8\textwidth]{diagrams/AllocationOverview24.png}
  \caption{Range of memory}
  \label{fig:RangeOfMemory}
\end{figure}
% \end{minipage}

% Ranges of memory are created based on bounds encoded to the FAT-Pointer based on the CHERI 128 bit 
% bounds compressed scheme. The Chuck between the specified upper and lower bounds are always physically
% contiguous. In the following implementation an arbitrary size huge page is initially allocated and within 
% this huge page custom size memory is allocated using a custom written mmap function which overwrites the existing block 
% based mmap function. Once memory is physically allocated using the custom mmap function call then bounds are set to it for tracking 
% the block of memory instead of using the TLB. 

In this implementation, memory ranges are established using bounds encoded within the FAT-pointer, adhering 
to the CHERI 128-bit bounds compression scheme\cite{woodruff_cheri_2019}. The memory chunk defined by the upper and lower bounds is 
always physically contiguous. Initially, a huge page of arbitrary size is allocated. Within this huge page, 
custom-sized memory segments are allocated using a custom-designed mmap function, which overrides the existing 
block-based mmap function. Once the memory is physically allocated through this custom mmap function, bounds 
are set to track the memory block, eliminating the need for traditional TLB usage for this purpose. Traditional TLB usage 
involves maintaining numerous TLB entries, often supplemented by an L2 TLB and other hierarchical structures, 
to translate virtual addresses to physical addresses. This approach requires multiple entries to handle various 
memory segments, leading to increased overhead and complexity in address translation. Conversely, 
the current approach streamlines this process by using a single TLB entry to translate multiple
 addresses within a contiguous memory range. This reduces the number of required TLB entries, 
 simplifying the translation process and improving efficiency. By consolidating address translations 
 into a single TLB entry, this method minimizes the overhead associated with managing numerous TLB entries 
 and leverages the bounds encoded within the FAT-pointer for efficient memory tracking and access. 
 This approach allows for precise and efficient memory management within the allocated huge page. 

%  The figure mentioned above demonstrates a simple use-case were the dark pink line refers to a huge page and 
%  the orange and blue lines are 2 equivalent to 2 Malloc calls allocating memory in different regions which in turn
%  emulates a block based memory allocator within a huge page using the bounds encoded in the FAT-Pointer.
\smallskip\noindent
Figure \ref{fig:RangeOfMemory} illustrates a straightforward use-case in which the dark pink line represents a single, 
large contiguous memory area, or huge page. Within this huge page, the orange and blue lines indicate 
two separate memory allocations equivalent to invoking malloc twice to allocate memory in distinct regions. 
This scenario simulates a block-based memory allocator operating within the confines of the huge page. 
The allocations leverage the bounds encoded in the FAT-pointer, ensuring tracking and efficient 
management of the allocated memory regions. By using the FAT-pointer bounds, this method maintains the 
integrity and contiguity of the allocated blocks within the huge page.

% This is done via the FreeBSD kernel Contiguous memory allocator. 

% \subsection{Fragmentation}
% % The problem with standard allocators which are physically contiguous
% % is that they are mostly on the
% todo
% \subsection{Allocation with huge pages}

\subsection{Software Stack}
\begin{figure}[h]
% \begin{minipage}[t]{0.4\linewidth}
\includegraphics[width=0.8\textwidth]{diagrams/SoftwareStack24.png}
\caption{Overview of the software stack}
\label{fig:SoftwareStack}
% \end{minipage}
\end{figure}

% The Software stack consists of CHERIBSD as the base operating system due to official 
% support by ARM to the Morello's performance counters. As mentioned in the figure above 
% there is a C program that is linked to the prototype memory allocator or allocators benchmarked 
% against as mentioned in section (x) as either a Shared object file at compile time or as a header 
% file for smaller memory allocators. The modified mmap function call which is designed to be physically 
% contiguous is linked to the contigmem driver which is modified from the DPDK library (reference).
% The contigmem driver is loaded on boot time and reserves arbitrary size of a huge page which is
% set based on the experiment conducted. 

The software stack is based on CHERIBSD, selected because ARM officially supports Morello's performance 
counters on this operating system. As illustrated in the figure \ref{fig:SoftwareStack}, the setup includes a C program that 
is linked to the prototype memory allocator or to various memory allocators being benchmarked, as described 
in section ("Evaluation section"). This linkage can occur in two ways: either as a shared object file during compile time 
for larger allocators, or as a header file for smaller allocators, ensuring flexibility and efficiency 
in memory management.

The custom mmap function, tailored to ensure physically contiguous memory allocation, is a key component 
of this system. This function is linked to the contigmem driver, which has been modified from the DPDK library 
to meet the specific needs of this implementation. The contigmem driver is essential for managing large contiguous 
memory blocks and is loaded during the system boot process. It reserves a huge page of arbitrary size, with the 
size parameter set based on the requirements of the conducted experiments.

This integration ensures that the memory allocation process is optimized for performance, leveraging the contiguity 
of memory blocks and the capabilities provided by the CHERI architecture and the Morello platform. By using the 
contigmem driver and the custom mmap function, the system achieves efficient memory allocation and tracking, 
crucial for the high-performance needs of the application.

\section{Evaluation}
To evaluate the performance of FAT-Pointer based range addresses a sample implementation we used
the morello board with CheriBSD's Benchmark ABI compilation mode for accurate performance recordings. 
The evaluation is to identify CheriBSD's default memory allocator SnMalloc against the prototype memory 
allocator using the contribution of this paper in terms of: 
To assess the performance of FAT-Pointer-based range addressing in the sample implementation, we utilized the 
Morello board running CheriBSD in Benchmark ABI compilation mode to ensure precise performance measurements. 
The evaluation focuses on comparing CheriBSD's default memory allocator, SnMalloc, against the prototype 
memory allocator developed in this study, with particular attention to the following aspects:

\begin{table}[!ht]
  \centering
  \begin{tabular}{|l|l|l|l|l|}
  \hline
      Metric name & type of graph & tool used & x axis & y axis \\ \hline
      DTLB L1 read & line graph & Pmcstat & Time & DTLB L1 reads (each second) \\ \hline
      DTLB L2 read & line graph & Pmcstat & Time & DTLB L2 reads (each second) \\ \hline
      DTLB walk & line graph & Pmcstat & Time & DTLB Walks (each second) \\ \hline
      L1 cache miss & line graph & Pmcstat & Time & L1 cache miss (each second) \\ \hline
      Wall clock run time & bar graph & time & Benchmarks & Time \\ \hline
  \end{tabular}
\end{table}

\subsection{Benchmarks used}:
To conduct the evaluations, we utilized the COZ benchmark suite, a well-regarded tool specifically designed 
to measure and analyze performance improvements in concurrent programs. The COZ benchmark suite provides a
robust framework for identifying bottlenecks and evaluating the performance impact of various optimization
techniques. By leveraging COZ, developers can gain precise insights into the efficiency and scalability of
their concurrent code, making it an ideal choice for rigorous performance analysis.

From the extensive set of benchmarks provided by COZ, we selected four representative C programs. 
These programs were chosen based on their relevance to common concurrent programming patterns and
their ability to effectively demonstrate the strengths and weaknesses of different optimization 
strategies. The selected programs cover a range of concurrency scenarios, ensuring a comprehensive
evaluation of performance improvements.

By implementing these modifications, we ensured that the selected C programs not only adhered to
CHERI's security model but also maintained their functional and performance characteristics. 
This allowed us to effectively use the COZ benchmark suite to analyze the performance in a CHERI-enhanced environment.

\begin{figure}[h]
  \includegraphics[width=0.8\textwidth]{diagrams/expirement-runs.png}
\end{figure}


% \begin{itemize} % Coz C programs
%   \item Kmeans
%   \item Histogram
%   \item Matrix multiply
% \end{itemize}


\subsection{Future work}
% The future plan is to transition from the current experiment, which involves working on the ARM architecture on the ARM Morello board. 
% The current limitation is that all memory reads must go through the TLB for translations. The future plan involves storing 
% the offset directly on the pointer and using the bounds in CHERI to enable block-based allocator behavior. This phase of the 
% experiment is intended to be conducted on the RISC-V implementation of CHERI, known as Tooba. 

% In the RISC-V implementation, the hardware Verilog design will be modified to allow bypassing the TLB. Once this concept is 
% integrated into the RISC-V Verilog implementation, the OS layer will be changed to a single-address-space operating system, 
% where there is no distinction between user space and kernel space. In this implementation, the kernel allocator will be the 
% same as the user space allocator since both can utilize a single contiguous chunk of memory.
The current experimental setup on the ARM Morello board is constrained by the requirement that all memory reads must 
pass through the Translation Lookaside Buffer (TLB) for address translation. This necessitates frequent TLB lookups, potentially 
leading to performance bottlenecks. The planned future work aims to address this by leveraging CHERI 
(Capability Hardware Enhanced RISC Instructions) extensions on the RISC-V architecture, specifically using the 
Tooba implementation.

\subsubsection{Storing Offsets Directly on Pointers}
In the current ARM Morello setup, address translations rely on the TLB.
The future approach on RISC-V Tooba involves storing the offset directly within the pointer. This is possible due to CHERI's capability model, which supports fine-grained memory protection and can encode bounds within pointers.
Utilizing Bounds in CHERI for Block-Based Allocation:

CHERI capabilities allow pointers to carry metadata about memory bounds, providing hardware-enforced memory safety.
By encoding the offset and bounds within the pointer, the system can directly access memory without needing intermediate translations via the TLB.
This enables the implementation of a block-based allocator that can efficiently manage memory allocations and deallocations within defined bounds.
Bypassing the TLB in RISC-V Tooba.
\subsubsection{Hardware Modifications}:
The Verilog design of the RISC-V processor will be modified to allow certain memory operations to bypass the TLB. This means that when a pointer with encoded offset and bounds is used, the system can directly compute the physical address from the capability information.
This modification reduces the dependency on the TLB, decreasing latency and improving performance, especially for frequent memory operations.
Transition to a Single-Address-Space Operating System (SASOS).
\subsubsection{Concept of SASOS}:
In traditional operating systems, there is a clear separation between user space and kernel space. This separation is enforced by memory protection mechanisms and address translation through the TLB.
In a Single-Address-Space Operating System, this distinction is removed. Both user applications and the kernel share the same contiguous address space.
\subsubsection{Advantages of SASOS with CHERI}:
\begin{itemize}
  \item Simplified Memory Management : Without the need to switch between user and kernel spaces, memory management becomes simpler and more efficient.
The kernel allocator can be the same as the user space allocator, operating on a single, contiguous chunk of memory.
  \item Unified Allocator: The unified memory allocator can efficiently manage memory for both kernel and user applications, leveraging CHERI's capability-based protection to prevent unauthorized access.
This reduces overhead and potential fragmentation issues associated with maintaining separate memory spaces.
\end{itemize}

% Explain contig module in detail 


% Graphs 

 

  % \begin{table}[!ht]
  %   \centering
  %   \begin{tabular}{|l|l|l|l|l|}
  %   \hline
  %       \textbf{Metric name} & \textbf{type of graph} & \textbf{tool used} & \textbf{x axis} & \textbf{y axis} \\ \hline
  %       DTLB L1 read & line graph & Pmcstat & Time & DTLB L1 reads (each second) \\ \hline
  %       DTLB L2 read & line graph & Pmcstat & Time & DTLB L2 reads (each second) \\ \hline
  %       DTLB walk & line graph & Pmcstat & Time & DTLB Walks (each second) \\ \hline
  %       L1 cache miss & line graph & Pmcstat & Time & L1 cache miss (each second) \\ \hline
  %       Wall clock run time & bar graph & time & Benchmarks & Time \\ \hline
  %       Resident memory usage & line graph & ps with rss & Time & Memory in MB \\ \hline
  %       Speed ups & bar graph & time & Benchmarks & Time \\ \hline
  %   \end{tabular}
  % \end{table}


  % \section{Limitations}

  % \section{Related work}


  % \section{Summary}






% \section{Introduction}
% ACM's consolidated article template, introduced in 2017, provides a
% consistent \LaTeX\ style for use across ACM publications, and
% incorporates accessibility and metadata-extraction functionality
% necessary for future Digital Library endeavors. Numerous ACM and
% SIG-specific \LaTeX\ templates have been examined, and their unique
% features incorporated into this single new template.

% If you are new to publishing with ACM, this document is a valuable
% guide to the process of preparing your work for publication. If you
% have published with ACM before, this document provides insight and
% instruction into more recent changes to the article template.

% The ``\verb|acmart|'' document class can be used to prepare articles
% for any ACM publication --- conference or journal, and for any stage
% of publication, from review to final ``camera-ready'' copy, to the
% author's own version, with {\itshape very} few changes to the source.

% \section{Template Overview}
% As noted in the introduction, the ``\verb|acmart|'' document class can
% be used to prepare many different kinds of documentation --- a
% double-anonymous initial submission of a full-length technical paper, a
% two-page SIGGRAPH Emerging Technologies abstract, a ``camera-ready''
% journal article, a SIGCHI Extended Abstract, and more --- all by
% selecting the appropriate {\itshape template style} and {\itshape
%   template parameters}.

% This document will explain the major features of the document
% class. For further information, the {\itshape \LaTeX\ User's Guide} is
% available from
% \url{https://www.acm.org/publications/proceedings-template}.

% \subsection{Template Styles}

% The primary parameter given to the ``\verb|acmart|'' document class is
% the {\itshape template style} which corresponds to the kind of publication
% or SIG publishing the work. This parameter is enclosed in square
% brackets and is a part of the {\verb|documentclass|} command:
% \begin{verbatim}
%   \documentclass[STYLE]{acmart}
% \end{verbatim}

% Journals use one of three template styles. All but three ACM journals
% use the {\verb|acmsmall|} template style:
% \begin{itemize}
% \item {\texttt{acmsmall}}: The default journal template style.
% \item {\texttt{acmlarge}}: Used by JOCCH and TAP.
% \item {\texttt{acmtog}}: Used by TOG.
% \end{itemize}

% The majority of conference proceedings documentation will use the {\verb|acmconf|} template style.
% \begin{itemize}
% \item {\texttt{sigconf}}: The default proceedings template style.
% \item{\texttt{sigchi}}: Used for SIGCHI conference articles.
% \item{\texttt{sigplan}}: Used for SIGPLAN conference articles.
% \end{itemize}

% \subsection{Template Parameters}

% In addition to specifying the {\itshape template style} to be used in
% formatting your work, there are a number of {\itshape template parameters}
% which modify some part of the applied template style. A complete list
% of these parameters can be found in the {\itshape \LaTeX\ User's Guide.}

% Frequently-used parameters, or combinations of parameters, include:
% \begin{itemize}
% \item {\texttt{anonymous,review}}: Suitable for a ``double-anonymous''
%   conference submission. Anonymizes the work and includes line
%   numbers. Use with the \texttt{\acmSubmissionID} command to print the
%   submission's unique ID on each page of the work.
% \item{\texttt{authorversion}}: Produces a version of the work suitable
%   for posting by the author.
% \item{\texttt{screen}}: Produces colored hyperlinks.
% \end{itemize}

% This document uses the following string as the first command in the
% source file:
% \begin{verbatim}
% \documentclass[acmsmall,screen,review]{acmart}
% \end{verbatim}

% \section{Modifications}

% Modifying the template --- including but not limited to: adjusting
% margins, typeface sizes, line spacing, paragraph and list definitions,
% and the use of the \verb|\vspace| command to manually adjust the
% vertical spacing between elements of your work --- is not allowed.

% {\bfseries Your document will be returned to you for revision if
%   modifications are discovered.}

% \section{Typefaces}

% The ``\verb|acmart|'' document class requires the use of the
% ``Libertine'' typeface family. Your \TeX\ installation should include
% this set of packages. Please do not substitute other typefaces. The
% ``\verb|lmodern|'' and ``\verb|ltimes|'' packages should not be used,
% as they will override the built-in typeface families.

% \section{Title Information}

% The title of your work should use capital letters appropriately -
% \url{https://capitalizemytitle.com/} has useful rules for
% capitalization. Use the {\verb|title|} command to define the title of
% your work. If your work has a subtitle, define it with the
% {\verb|subtitle|} command.  Do not insert line breaks in your title.

% If your title is lengthy, you must define a short version to be used
% in the page headers, to prevent overlapping text. The \verb|title|
% command has a ``short title'' parameter:
% \begin{verbatim}
%   \title[short title]{full title}
% \end{verbatim}

% \section{Authors and Affiliations}

% Each author must be defined separately for accurate metadata
% identification.  As an exception, multiple authors may share one
% affiliation. Authors' names should not be abbreviated; use full first
% names wherever possible. Include authors' e-mail addresses whenever
% possible.

% Grouping authors' names or e-mail addresses, or providing an ``e-mail
% alias,'' as shown below, is not acceptable:
% \begin{verbatim}
%   \author{Brooke Aster, David Mehldau}
%   \email{dave,judy,steve@university.edu}
%   \email{firstname.lastname@phillips.org}
% \end{verbatim}

% The \verb|authornote| and \verb|authornotemark| commands allow a note
% to apply to multiple authors --- for example, if the first two authors
% of an article contributed equally to the work.

% If your author list is lengthy, you must define a shortened version of
% the list of authors to be used in the page headers, to prevent
% overlapping text. The following command should be placed just after
% the last \verb|\author{}| definition:
% \begin{verbatim}
%   \renewcommand{\shortauthors}{McCartney, et al.}
% \end{verbatim}
% Omitting this command will force the use of a concatenated list of all
% of the authors' names, which may result in overlapping text in the
% page headers.

% The article template's documentation, available at
% \url{https://www.acm.org/publications/proceedings-template}, has a
% complete explanation of these commands and tips for their effective
% use.

% Note that authors' addresses are mandatory for journal articles.

% \section{Rights Information}

% Authors of any work published by ACM will need to complete a rights
% form. Depending on the kind of work, and the rights management choice
% made by the author, this may be copyright transfer, permission,
% license, or an OA (open access) agreement.

% Regardless of the rights management choice, the author will receive a
% copy of the completed rights form once it has been submitted. This
% form contains \LaTeX\ commands that must be copied into the source
% document. When the document source is compiled, these commands and
% their parameters add formatted text to several areas of the final
% document:
% \begin{itemize}
% \item the ``ACM Reference Format'' text on the first page.
% \item the ``rights management'' text on the first page.
% \item the conference information in the page header(s).
% \end{itemize}

% Rights information is unique to the work; if you are preparing several
% works for an event, make sure to use the correct set of commands with
% each of the works.

% The ACM Reference Format text is required for all articles over one
% page in length, and is optional for one-page articles (abstracts).

% \section{CCS Concepts and User-Defined Keywords}

% Two elements of the ``acmart'' document class provide powerful
% taxonomic tools for you to help readers find your work in an online
% search.

% The ACM Computing Classification System ---
% \url{https://www.acm.org/publications/class-2012} --- is a set of
% classifiers and concepts that describe the computing
% discipline. Authors can select entries from this classification
% system, via \url{https://dl.acm.org/ccs/ccs.cfm}, and generate the
% commands to be included in the \LaTeX\ source.

% User-defined keywords are a comma-separated list of words and phrases
% of the authors' choosing, providing a more flexible way of describing
% the research being presented.

% CCS concepts and user-defined keywords are required for for all
% articles over two pages in length, and are optional for one- and
% two-page articles (or abstracts).

% \section{Sectioning Commands}

% Your work should use standard \LaTeX\ sectioning commands:
% \verb|section|, \verb|subsection|, \verb|subsubsection|, and
% \verb|paragraph|. They should be numbered; do not remove the numbering
% from the commands.

% Simulating a sectioning command by setting the first word or words of
% a paragraph in boldface or italicized text is {\bfseries not allowed.}

% \section{Tables}

% The ``\verb|acmart|'' document class includes the ``\verb|booktabs|''
% package --- \url{https://ctan.org/pkg/booktabs} --- for preparing
% high-quality tables.

% Table captions are placed {\itshape above} the table.

% Because tables cannot be split across pages, the best placement for
% them is typically the top of the page nearest their initial cite.  To
% ensure this proper ``floating'' placement of tables, use the
% environment \textbf{table} to enclose the table's contents and the
% table caption.  The contents of the table itself must go in the
% \textbf{tabular} environment, to be aligned properly in rows and
% columns, with the desired horizontal and vertical rules.  Again,
% detailed instructions on \textbf{tabular} material are found in the
% \textit{\LaTeX\ User's Guide}.

% Immediately following this sentence is the point at which
% Table~\ref{tab:freq} is included in the input file; compare the
% placement of the table here with the table in the printed output of
% this document.

% \begin{table}
%   \caption{Frequency of Special Characters}
%   \label{tab:freq}
%   \begin{tabular}{ccl}
%     \toprule
%     Non-English or Math&Frequency&Comments\\
%     \midrule
%     \O & 1 in 1,000& For Swedish names\\
%     $\pi$ & 1 in 5& Common in math\\
%     \$ & 4 in 5 & Used in business\\
%     $\Psi^2_1$ & 1 in 40,000& Unexplained usage\\
%   \bottomrule
% \end{tabular}
% \end{table}

% To set a wider table, which takes up the whole width of the page's
% live area, use the environment \textbf{table*} to enclose the table's
% contents and the table caption.  As with a single-column table, this
% wide table will ``float'' to a location deemed more
% desirable. Immediately following this sentence is the point at which
% Table~\ref{tab:commands} is included in the input file; again, it is
% instructive to compare the placement of the table here with the table
% in the printed output of this document.

% \begin{table*}
%   \caption{Some Typical Commands}
%   \label{tab:commands}
%   \begin{tabular}{ccl}
%     \toprule
%     Command &A Number & Comments\\
%     \midrule
%     \texttt{{\char'134}author} & 100& Author \\
%     \texttt{{\char'134}table}& 300 & For tables\\
%     \texttt{{\char'134}table*}& 400& For wider tables\\
%     \bottomrule
%   \end{tabular}
% \end{table*}

% Always use midrule to separate table header rows from data rows, and
% use it only for this purpose. This enables assistive technologies to
% recognise table headers and support their users in navigating tables
% more easily.

% \section{Math Equations}
% You may want to display math equations in three distinct styles:
% inline, numbered or non-numbered display.  Each of the three are
% discussed in the next sections.

% \subsection{Inline (In-text) Equations}
% A formula that appears in the running text is called an inline or
% in-text formula.  It is produced by the \textbf{math} environment,
% which can be invoked with the usual
% \texttt{{\char'134}begin\,\ldots{\char'134}end} construction or with
% the short form \texttt{\$\,\ldots\$}. You can use any of the symbols
% and structures, from $\alpha$ to $\omega$, available in
% \LaTeX~\cite{Lamport:LaTeX}; this section will simply show a few
% examples of in-text equations in context. Notice how this equation:
% \begin{math}
%   \lim_{n\rightarrow \infty}x=0
% \end{math},
% set here in in-line math style, looks slightly different when
% set in display style.  (See next section).

% \subsection{Display Equations}
% A numbered display equation---one set off by vertical space from the
% text and centered horizontally---is produced by the \textbf{equation}
% environment. An unnumbered display equation is produced by the
% \textbf{displaymath} environment.

% Again, in either environment, you can use any of the symbols and
% structures available in \LaTeX\@; this section will just give a couple
% of examples of display equations in context.  First, consider the
% equation, shown as an inline equation above:
% \begin{equation}
%   \lim_{n\rightarrow \infty}x=0
% \end{equation}
% Notice how it is formatted somewhat differently in
% the \textbf{displaymath}
% environment.  Now, we'll enter an unnumbered equation:
% \begin{displaymath}
%   \sum_{i=0}^{\infty} x + 1
% \end{displaymath}
% and follow it with another numbered equation:
% \begin{equation}
%   \sum_{i=0}^{\infty}x_i=\int_{0}^{\pi+2} f
% \end{equation}
% just to demonstrate \LaTeX's able handling of numbering.

% \section{Figures}

% The ``\verb|figure|'' environment should be used for figures. One or
% more images can be placed within a figure. If your figure contains
% third-party material, you must clearly identify it as such, as shown
% in the example below.
% \begin{figure}[h]
%   \centering
%   \includegraphics[width=\linewidth]{sample-franklin}
%   \caption{1907 Franklin Model D roadster. Photograph by Harris \&
%     Ewing, Inc. [Public domain], via Wikimedia
%     Commons. (\url{https://goo.gl/VLCRBB}).}
%   \Description{A woman and a girl in white dresses sit in an open car.}
% \end{figure}

% Your figures should contain a caption which describes the figure to
% the reader.

% Figure captions are placed {\itshape below} the figure.

% Every figure should also have a figure description unless it is purely
% decorative. These descriptions convey what’s in the image to someone
% who cannot see it. They are also used by search engine crawlers for
% indexing images, and when images cannot be loaded.

% A figure description must be unformatted plain text less than 2000
% characters long (including spaces).  {\bfseries Figure descriptions
%   should not repeat the figure caption – their purpose is to capture
%   important information that is not already provided in the caption or
%   the main text of the paper.} For figures that convey important and
% complex new information, a short text description may not be
% adequate. More complex alternative descriptions can be placed in an
% appendix and referenced in a short figure description. For example,
% provide a data table capturing the information in a bar chart, or a
% structured list representing a graph.  For additional information
% regarding how best to write figure descriptions and why doing this is
% so important, please see
% \url{https://www.acm.org/publications/taps/describing-figures/}.

% \subsection{The ``Teaser Figure''}

% A ``teaser figure'' is an image, or set of images in one figure, that
% are placed after all author and affiliation information, and before
% the body of the article, spanning the page. If you wish to have such a
% figure in your article, place the command immediately before the
% \verb|\maketitle| command:
% \begin{verbatim}
%   \begin{teaserfigure}
%     \includegraphics[width=\textwidth]{sampleteaser}
%     \caption{figure caption}
%     \Description{figure description}
%   \end{teaserfigure}
% \end{verbatim}

% \section{Citations and Bibliographies}

% The use of \BibTeX\ for the preparation and formatting of one's
% references is strongly recommended. Authors' names should be complete
% --- use full first names (``Donald E. Knuth'') not initials
% (``D. E. Knuth'') --- and the salient identifying features of a
% reference should be included: title, year, volume, number, pages,
% article DOI, etc.

% The bibliography is included in your source document with these two
% commands, placed just before the \verb|\end{document}| command:
% \begin{verbatim}
%   \bibliographystyle{ACM-Reference-Format}
%   \bibliography{bibfile}
% \end{verbatim}
% where ``\verb|bibfile|'' is the name, without the ``\verb|.bib|''
% suffix, of the \BibTeX\ file.

% Citations and references are numbered by default. A small number of
% ACM publications have citations and references formatted in the
% ``author year'' style; for these exceptions, please include this
% command in the {\bfseries preamble} (before the command
% ``\verb|\begin{document}|'') of your \LaTeX\ source:
% \begin{verbatim}
%   \citestyle{acmauthoryear}
% \end{verbatim}


  % Some examples.  A paginated journal article \cite{Abril07}, an
  % enumerated journal article \cite{Cohen07}, a reference to an entire
  % issue \cite{JCohen96}, a monograph (whole book) \cite{Kosiur01}, a
  % monograph/whole book in a series (see 2a in spec. document)
  % \cite{Harel79}, a divisible-book such as an anthology or compilation
  % \cite{Editor00} followed by the same example, however we only output
  % the series if the volume number is given \cite{Editor00a} (so
  % Editor00a's series should NOT be present since it has no vol. no.),
  % a chapter in a divisible book \cite{Spector90}, a chapter in a
  % divisible book in a series \cite{Douglass98}, a multi-volume work as
  % book \cite{Knuth97}, a couple of articles in a proceedings (of a
  % conference, symposium, workshop for example) (paginated proceedings
  % article) \cite{Andler79, Hagerup1993}, a proceedings article with
  % all possible elements \cite{Smith10}, an example of an enumerated
  % proceedings article \cite{VanGundy07}, an informally published work
  % \cite{Harel78}, a couple of preprints \cite{Bornmann2019,
  %   AnzarootPBM14}, a doctoral dissertation \cite{Clarkson85}, a
  % master's thesis: \cite{anisi03}, an online document / world wide web
  % resource \cite{Thornburg01, Ablamowicz07, Poker06}, a video game
  % (Case 1) \cite{Obama08} and (Case 2) \cite{Novak03} and \cite{Lee05}
  % and (Case 3) a patent \cite{JoeScientist001}, work accepted for
  % publication \cite{rous08}, 'YYYYb'-test for prolific author
  % \cite{SaeediMEJ10} and \cite{SaeediJETC10}. Other cites might
  % contain 'duplicate' DOI and URLs (some SIAM articles)
  % \cite{Kirschmer:2010:AEI:1958016.1958018}. Boris / Barbara Beeton:
  % multi-volume works as books \cite{MR781536} and \cite{MR781537}. A
  % couple of citations with DOIs:
  % \cite{2004:ITE:1009386.1010128,Kirschmer:2010:AEI:1958016.1958018}. Online
  % citations: \cite{TUGInstmem, Thornburg01, CTANacmart}.
  % Artifacts: \cite{R} and \cite{UMassCitations}.

% \section{Acknowledgments}

% Identification of funding sources and other support, and thanks to
% individuals and groups that assisted in the research and the
% preparation of the work should be included in an acknowledgment
% section, which is placed just before the reference section in your
% document.

% This section has a special environment:
% \begin{verbatim}
%   \begin{acks}
%   ...
%   \end{acks}
% \end{verbatim}
% so that the information contained therein can be more easily collected
% during the article metadata extraction phase, and to ensure
% consistency in the spelling of the section heading.

% Authors should not prepare this section as a numbered or unnumbered {\verb|\section|}; please use the ``{\verb|acks|}'' environment.

% \section{Appendices}

% If your work needs an appendix, add it before the
% ``\verb|\end{document}|'' command at the conclusion of your source
% document.

% Start the appendix with the ``\verb|appendix|'' command:
% \begin{verbatim}
%   \appendix
% \end{verbatim}
% and note that in the appendix, sections are lettered, not
% numbered. This document has two appendices, demonstrating the section
% and subsection identification method.

% \section{Multi-language papers}

% Papers may be written in languages other than English or include
% titles, subtitles, keywords and abstracts in different languages (as a
% rule, a paper in a language other than English should include an
% English title and an English abstract).  Use \verb|language=...| for
% every language used in the paper.  The last language indicated is the
% main language of the paper.  For example, a French paper with
% additional titles and abstracts in English and German may start with
% the following command
% \begin{verbatim}
% \documentclass[sigconf, language=english, language=german,
%                language=french]{acmart}
% \end{verbatim}

% The title, subtitle, keywords and abstract will be typeset in the main
% language of the paper.  The commands \verb|\translatedXXX|, \verb|XXX|
% begin title, subtitle and keywords, can be used to set these elements
% in the other languages.  The environment \verb|translatedabstract| is
% used to set the translation of the abstract.  These commands and
% environment have a mandatory first argument: the language of the
% second argument.  See \verb|sample-sigconf-i13n.tex| file for examples
% of their usage.

% \section{SIGCHI Extended Abstracts}

% The ``\verb|sigchi-a|'' template style (available only in \LaTeX\ and
% not in Word) produces a landscape-orientation formatted article, with
% a wide left margin. Three environments are available for use with the
% ``\verb|sigchi-a|'' template style, and produce formatted output in
% the margin:
% \begin{description}
% \item[\texttt{sidebar}:]  Place formatted text in the margin.
% \item[\texttt{marginfigure}:] Place a figure in the margin.
% \item[\texttt{margintable}:] Place a table in the margin.
% \end{description}

%%
%% The acknowledgments section is defined using the "acks" environment
%% (and NOT an unnumbered section). This ensures the proper
%% identification of the section in the article metadata, and the
%% consistent spelling of the heading.
% \begin{acks}
% To Robert, for the bagels and explaining CMYK and color spaces.
% \end{acks}

% \begin{acks}
%   To Robert, for the bagels and explaining CMYK and color spaces.
% \end{acks}

%%
%% The next two lines define the bibliography style to be used, and
%% the bibliography file.
\bibliographystyle{ACM-Reference-Format}
\bibliography{references}

% \appendix

% \section{Research Methods}

% \subsection{Part One}

% Lorem ipsum dolor sit amet, consectetur adipiscing elit. Morbi
% malesuada, quam in pulvinar varius, metus nunc fermentum urna, id
% sollicitudin purus odio sit amet enim. Aliquam ullamcorper eu ipsum
% vel mollis. Curabitur quis dictum nisl. Phasellus vel semper risus, et
% lacinia dolor. Integer ultricies commodo sem nec semper.

% \subsection{Part Two}

% Etiam commodo feugiat nisl pulvinar pellentesque. Etiam auctor sodales
% ligula, non varius nibh pulvinar semper. Suspendisse nec lectus non
% ipsum convallis congue hendrerit vitae sapien. Donec at laoreet
% eros. Vivamus non purus placerat, scelerisque diam eu, cursus
% ante. Etiam aliquam tortor auctor efficitur mattis.

% \section{Online Resources}

% Nam id fermentum dui. Suspendisse sagittis tortor a nulla mollis, in
% pulvinar ex pretium. Sed interdum orci quis metus euismod, et sagittis
% enim maximus. Vestibulum gravida massa ut felis suscipit
% congue. Quisque mattis elit a risus ultrices commodo venenatis eget
% dui. Etiam sagittis eleifend elementum.

% Nam interdum magna at lectus dignissim, ac dignissim lorem
% rhoncus. Maecenas eu arcu ac neque placerat aliquam. Nunc pulvinar
% massa et mattis lacinia.


\end{document}
\endinput
%%
%% End of file `sample-acmsmall-submission.tex'.
