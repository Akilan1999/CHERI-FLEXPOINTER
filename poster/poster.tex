\documentclass[25pt, a0paper, portrait]{tikzposter}
\usetikzlibrary{fit,arrows,tikzmark,shadows,calc,automata}
\newcommand\NameBlock[1]{\node[fit=(blockbody)(blocktitle),inner sep=5pt] (#1) {};}
\usepackage[utf8]{inputenc}
\usepackage{subcaption}
\usepackage{adjustbox}
\usepackage{booktabs}

\usepackage[backend=bibtex]{biblatex}
\addbibresource{poster.bib}

\usepackage{authblk}
\makeatletter
% revert \maketitle to its old definition
% \def\maketitle{\AB@maketitle}
\renewcommand\maketitle{\AB@maketitle}

% put affiliations into one line
\renewcommand\AB@affilsepx{\quad\protect\Affilfont}
%
\makeatother
% set font for affiliations
\renewcommand\Affilfont{\Large}


\tikzposterlatexaffectionproofoff

% \usepackage{poster-style}
\usepackage{filecontents}
\usepackage{listings}


\title{FAT-Pointer based range TLB}
% \author{}
\date{\today}
% \institute{Heriot-Watt University UK \and Université de Toulouse France,
% and Carleton University Canada}

\author[1]{\Large Akilan Selvacoumar}
\author[1]{ Robert Stewart}
\author[1]{ Hans-Wolfgang Loidl}
\author[1]{ Ryad Soobhany}
\affil[1]{Heriot-Watt University, UK}

% \titlegraphic{
% \includegraphics[width=0.1\paperwidth]{hw-logo.png}
% }

\usepackage{blindtext}
\usepackage{comment}

% Default, Rays, Basic, Simple, Envelope, Wave, Board, Autumn, Desert
% 
% good: Default, Basic, Simple
\usetheme{Basic}

% \renewcommand{\titleposright}{50mm}
% \titleposright=100mm
\definetitlestyle{sampletitle}{
  innersep=-10pt,
  titletotopverticalspace=-400mm,
  titletoblockverticalspace=10mm
}
{}

\usetitlestyle{sampletitle}
% \titleposleft=-300mm

% \usetitlestyle[titletotopverticalspace=-50mm,]{Basic}


% Britain, Australia, Sweden, Spain,Russia,Denmark,Germany
\usecolorstyle{Default}



\begin{document}
%   SICSA Postdoctoral and Early Career Researcher Exchanges.};

% removes space at top
\maketitle[titletotopverticalspace=-5mm]
% \maketitle

% \block{~}
% {
% \blindtext
% }

\begin{columns}
  \column{0.5}

  \block{Abstract}{

    % \begin{minipage}[t]{0.6\linewidth}

      \begin{tikzfigure}
        % \includegraphics[width=0.4\textwidth]{images/abstraction-poster.jpg}
      \end{tikzfigure}
      Unikernels are an interesting approach to improve performance by using a slimmed down kernel 
      for a specific task. The following paper benchmarks the parallel performance of unikernels,
i.e in a cloud based scenario. The results will be compared against the
same application running on a docker container and a monolithic OS as
well



    \vspace{0.5cm}
  }

  \block{Research Questions and Goals}{

    \hspace{0.5cm}

    \begin{minipage}[t]{0.8\linewidth}
      \textbf{\textit{Research Questions:}}
      \begin{itemize}
        \item (RQ1) Can unikernels be specialised for parallel programs?
        % \item (RQ2) Would Unikernels running parallel programs out-perform cloud based environment 
        % monolithic OS or containers in terms of wall-clock runtimes, CPU
        % profiling and memory profiling?
      % \item \textit{parallelising actors can increase throughput}
      \end{itemize} 

    \end{minipage}%
    \hspace{0.2cm}

    \vspace{0.1cm}

    \begin{tikzfigure}
      % \includegraphics[width=0.4\textwidth]{images/parallel-layers-poster.png}
    \end{tikzfigure}

    \vspace{-0.5cm}
  }
  
  \block{Expirement Proposal}{

    \begin{minipage}[t]{0.4\linewidth}
      \vspace{-0.9cm}
      \begin{tikzfigure}{}
        %  \includegraphics[width=2.53\textwidth]{Mandelbrot.png}
      \end{tikzfigure}
    \end{minipage}

    \vspace{0.6cm}

    % The following section describes the strategies used to measure distributed map-reduce applications on unikernels.
    % The metrics and scenarios measured would help towards the short-term goal of understanding the benefits of unikernels with
    % distributed map reduce. 
    \hspace{0.5cm}
    \begin{minipage}[t]{0.8\linewidth}

      % \begin{itemize}
      % % \item \textbf{\textit{Benchmark application}}
      % %   \begin{itemize}
      % %   \item \textit{Mandelbrot}: The Mandelbrot Go implementation was used to
      % %   benchmark parallelised applications on Uni-kernels. The implementation uses
      % %   Go routines to spawn multiple threads. The parts parallelised of the Mandelbrot
      % %   implementation was the render part ,particularly the Mandelbrot iteration and
      % %   Linear interpolation.
      % %   \end{itemize}
      % % \item \textit{parallelising actors can increase throughput}
      % \end{itemize}

      % \begin{itemize}
      %   \item \textbf{\textit{Comparators}}
      %     \begin{itemize}
      %     \item \textit{Unikernel}
      %     \item \textit{Monolithic OS}
      %     \item \textit{Docker Container}
      %     \end{itemize}
      %   % \item \textit{parallelising actors can increase throughput}
      %   \end{itemize}

        % \begin{itemize}
        %   \item \textbf{\textit{Benchmark Metrics}}
        %     \begin{itemize}
        %     \item \textit{Boot-up time}
        %     \item \textit{Wall clock run times}
        %     \item \textit{Parallel Speed ups}
        %     \item \textit{Parallel efficiency}
        %     \end{itemize}
        %   % \item \textit{parallelising actors can increase throughput}
        %   \end{itemize}

    \end{minipage}%
    \hspace{0.2cm}

    \vspace{0.1cm}

    \begin{tikzfigure}
      % \includegraphics[width=0.4\textwidth]{images/parallel-layers-poster.png}
    \end{tikzfigure}

    \vspace{-0.5cm}
  }

  \NameBlock{transformation}
  \block{Exploiting CHERI pointers}{
    \hspace{0.4cm}
    % Bootup-times
    \begin{minipage}[t]{0.4\linewidth}
      \vspace{-0.9cm}
      \begin{tikzfigure}{}
        %  \includegraphics[width=2.47\textwidth]{Bootup-times.png}
      \end{tikzfigure}
    \end{minipage}



 }

    \NameBlock{transformation}
    \block{Future Work}{
      \hspace{0.4cm}

     \begin{itemize}
       \item Building a parallel benchmark suite for Unikernels.
       \item Analysing the metrics provided by the Go compiler such as Heap usage, Number OS threads created by run time etc…
       \item Benchmarking other Unikernel implementation using the benchmark suite (1) 
     \end{itemize}

 

   }

  \column{0.5}


  % \NameBlock{results}
  % \block{Experimental Setup (Mandelbrot program)}{
  %    \hspace{0.4cm}

  %   \begin{minipage}[t]{0.4\linewidth}
  %     \vspace{-0.9cm}
  %     \begin{tikzfigure}{}
  %       \includegraphics[width=2.54\textwidth]{expirement.png}
  %     \end{tikzfigure}
  %   \end{minipage}

  %   \vspace{0.4cm}

  %   \begin{itemize}
  %     \item \textit{Scenario 1\/}: Height of 1000 and 3000 iterations.
  %     \item \textit{Scenario 2\/}: Height of 2000 and 6000 iterations.
  %   \end{itemize}

  %   % \hspace{0.2cm}

  %   \begin{minipage}[t]{0.8\linewidth}

  %   \end{minipage}%

  %   \begin{tikzfigure}
  %     % \includegraphics[width=0.4\textwidth]{images/parallel-layers-poster.png}
  %   \end{tikzfigure}

  %   \vspace{-2.5cm}
  % }

%   \block{Results}{
%     \vspace{0.4cm}

%    \begin{minipage}[t]{0.4\linewidth}
%     %  \hspace{-0.9cm}
%      \textbf{\textit{Wall clock run times}}
%      \begin{tikzfigure}{}
%        \includegraphics[width=2.54\textwidth]{WallClockRunTimes.png}
%      \end{tikzfigure}
%    \end{minipage}
%    \hspace{0.8cm}
   
%    \vspace{0.8cm}
%    \begin{minipage}[t]{0.4\linewidth}
%     % \vspace{-0.9cm}
%     \textbf{\textit{Parallel Speedups}}
%     \begin{tikzfigure}{}
%       \includegraphics[width=2.54\textwidth]{ParallelSpeedups.png}
%     \end{tikzfigure}
%   \end{minipage}

%   \vspace{0.8cm}
%   \begin{minipage}[t]{0.4\linewidth}
%     % \vspace{-0.9cm}
%     \textbf{\textit{Parallel Efficiency}}
%     \begin{tikzfigure}{}
%       \includegraphics[width=2.54\textwidth]{ParallelEffciency.png}
%     \end{tikzfigure}
%   \end{minipage}

%    % \hspace{0.2cm}

%    \begin{minipage}[t]{0.8\linewidth}

%    \end{minipage}%

%    \begin{tikzfigure}
%      % \includegraphics[width=0.4\textwidth]{images/parallel-layers-poster.png}
%    \end{tikzfigure}

%    \vspace{-2.5cm}
%  }
  
    % \NameBlock{model-checking}
    
  % \block{Conclusion}{
  %   \vspace{0.3cm}
  %   The empirical evidence gained from
  %   these measurements, running two different scenarios, will be used
  %   to answer the three research questions stated in the introduction
  %   of the paper. The empirical data from running the experiments on
  %   unikernels will provide a better understanding on how unikernels
  %   perform on a distributed memory environments.

  %   \vspace{1cm}

  %   \mbox{}\vspace{-\baselineskip}

  %   \printbibliography[heading=none]

  % }


\end{columns}

\begin{columns}
  \column{0.5}


    
  \column{0.5}
  
  \end{columns}

\end{document}